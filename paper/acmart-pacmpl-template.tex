%% For double-blind review submission, w/o CCS and ACM Reference (max submission space)
\documentclass[acmsmall,review,anonymous]{acmart}\settopmatter{printfolios=true,printccs=false,printacmref=false}
%% For double-blind review submission, w/ CCS and ACM Reference
%\documentclass[acmsmall,review,anonymous]{acmart}\settopmatter{printfolios=true}
%% For single-blind review submission, w/o CCS and ACM Reference (max submission space)
%\documentclass[acmsmall,review]{acmart}\settopmatter{printfolios=true,printccs=false,printacmref=false}
%% For single-blind review submission, w/ CCS and ACM Reference
%\documentclass[acmsmall,review]{acmart}\settopmatter{printfolios=true}
%% For final camera-ready submission, w/ required CCS and ACM Reference
%\documentclass[acmsmall]{acmart}\settopmatter{}


%% Journal information
%% Supplied to authors by publisher for camera-ready submission;
%% use defaults for review submission.
\acmJournal{PACMPL}
\acmVolume{1}
\acmNumber{CONF} % CONF = POPL or ICFP or OOPSLA
\acmArticle{1}
\acmYear{2018}
\acmMonth{1}
\acmDOI{} % \acmDOI{10.1145/nnnnnnn.nnnnnnn}
\startPage{1}

%% Copyright information
%% Supplied to authors (based on authors' rights management selection;
%% see authors.acm.org) by publisher for camera-ready submission;
%% use 'none' for review submission.
\setcopyright{none}
%\setcopyright{acmcopyright}
%\setcopyright{acmlicensed}
%\setcopyright{rightsretained}
%\copyrightyear{2018}           %% If different from \acmYear

%% Bibliography style
\bibliographystyle{ACM-Reference-Format}
%% Citation style
%% Note: author/year citations are required for papers published as an
%% issue of PACMPL.
\citestyle{acmauthoryear}   %% For author/year citations


%%%%%%%%%%%%%%%%%%%%%%%%%%%%%%%%%%%%%%%%%%%%%%%%%%%%%%%%%%%%%%%%%%%%%%
%% Note: Authors migrating a paper from PACMPL format to traditional
%% SIGPLAN proceedings format must update the '\documentclass' and
%% topmatter commands above; see 'acmart-sigplanproc-template.tex'.
%%%%%%%%%%%%%%%%%%%%%%%%%%%%%%%%%%%%%%%%%%%%%%%%%%%%%%%%%%%%%%%%%%%%%%


%% Some recommended packages.
\usepackage{booktabs}   %% For formal tables:
                        %% http://ctan.org/pkg/booktabs
\usepackage{subcaption} %% For complex figures with subfigures/subcaptions
                        %% http://ctan.org/pkg/subcaption


\begin{document}

%% Title information
\title[Short Title]{Full Title}         %% [Short Title] is optional;
                                        %% when present, will be used in
                                        %% header instead of Full Title.
\titlenote{with title note}             %% \titlenote is optional;
                                        %% can be repeated if necessary;
                                        %% contents suppressed with 'anonymous'
\subtitle{Subtitle}                     %% \subtitle is optional
\subtitlenote{with subtitle note}       %% \subtitlenote is optional;
                                        %% can be repeated if necessary;
                                        %% contents suppressed with 'anonymous'


%% Author information
%% Contents and number of authors suppressed with 'anonymous'.
%% Each author should be introduced by \author, followed by
%% \authornote (optional), \orcid (optional), \affiliation, and
%% \email.
%% An author may have multiple affiliations and/or emails; repeat the
%% appropriate command.
%% Many elements are not rendered, but should be provided for metadata
%% extraction tools.

%% Author with single affiliation.
\author{First1 Last1}
\authornote{with author1 note}          %% \authornote is optional;
                                        %% can be repeated if necessary
\orcid{nnnn-nnnn-nnnn-nnnn}             %% \orcid is optional
\affiliation{
  \position{Position1}
  \department{Department1}              %% \department is recommended
  \institution{Institution1}            %% \institution is required
  \streetaddress{Street1 Address1}
  \city{City1}
  \state{State1}
  \postcode{Post-Code1}
  \country{Country1}                    %% \country is recommended
}
\email{first1.last1@inst1.edu}          %% \email is recommended

%% Author with two affiliations and emails.
\author{First2 Last2}
\authornote{with author2 note}          %% \authornote is optional;
                                        %% can be repeated if necessary
\orcid{nnnn-nnnn-nnnn-nnnn}             %% \orcid is optional
\affiliation{
  \position{Position2a}
  \department{Department2a}             %% \department is recommended
  \institution{Institution2a}           %% \institution is required
  \streetaddress{Street2a Address2a}
  \city{City2a}
  \state{State2a}
  \postcode{Post-Code2a}
  \country{Country2a}                   %% \country is recommended
}
\email{first2.last2@inst2a.com}         %% \email is recommended
\affiliation{
  \position{Position2b}
  \department{Department2b}             %% \department is recommended
  \institution{Institution2b}           %% \institution is required
  \streetaddress{Street3b Address2b}
  \city{City2b}
  \state{State2b}
  \postcode{Post-Code2b}
  \country{Country2b}                   %% \country is recommended
}
\email{first2.last2@inst2b.org}         %% \email is recommended


%% Abstract
%% Note: \begin{abstract}...\end{abstract} environment must come
%% before \maketitle command
\begin{abstract}
Transformations and musical patterns -> Haskell -> 
\end{abstract}


%% 2012 ACM Computing Classification System (CSS) concepts
%% Generate at 'http://dl.acm.org/ccs/ccs.cfm'.
\begin{CCSXML}
<ccs2012>
<concept>
<concept_id>10011007.10011006.10011008</concept_id>
<concept_desc>Software and its engineering~General programming languages</concept_desc>
<concept_significance>500</concept_significance>
</concept>
<concept>
<concept_id>10003456.10003457.10003521.10003525</concept_id>
<concept_desc>Social and professional topics~History of programming languages</concept_desc>
<concept_significance>300</concept_significance>
</concept>
</ccs2012>
\end{CCSXML}

\ccsdesc[500]{Software and its engineering~General programming languages}
\ccsdesc[300]{Social and professional topics~History of programming languages}
%% End of generated code


%% Keywords
%% comma separated list
\keywords{transformation, edit distance, musical patterns, evaluation,
  clustering, ...}  %% \keywords are mandatory in final camera-ready submission


%% \maketitle
%% Note: \maketitle command must come after title commands, author
%% commands, abstract environment, Computing Classification System
%% environment and commands, and keywords command.
\maketitle


\section{Introduction}

%\paragraph{MIR and functional programming}
%DSL
%Type
%Many possibility to explore.

\paragraph{Hierachical structures in music}
Music is known to have rich hierachical structures, from form to phrase,
harmonic structures, melodic constructs, etc.. One can comprehend music at
different time scales. There have been many theories on this topic, such as the
General Theory of Tonal Music (GTTM) \cite{} and the Schekerian theory of melodic reduction
\cite{}. By examining what is the backbone of the piece and where
are the rest, we can view music from different levels of importance and details.

\paragraph{Music Information Retrieval (MIR)}
In the research area of MIR, with understandings in the hierachical structures in music, many
useful tools were made: musical analysis assistant \cite{}, compositional tools
\cite{} and information retrieval systems \cite{} just to list a few.
There have been also much research on how one could understand, represent and extract the
hierachical structures automatically. A balance and feedback loop between the theories and the
applications have stimulated much interests in the topic of hierachies in music. 

\paragraph{Metrical structures}
Metrical structure plays an important role in the construction and the
perception of the hierachical structures in music. Depending on the locations of
the musical events on the metrical grid or their relational positions to other
notes, one can assign metrical weights/importance to the notes. The notes at more important metrical
positions form the anchors in the hierachical structure. 

\paragraph{Fractal Geometry}
Fractal geometry is an established area of mathematics. The box-counting way of
calculating the fractal dimensions are known to be able to measure the roughness
of contours. For example, the fractal dimension of the coastline of the United
Kingdom is measured to be... and the ... for Sweden. It has been used in time
series analysis, dynamical system, and where there is self-similarity in
general. 

Zooming-in/out: inspired by fractal dimension calculation, going in-between
hierachies, consider different levels of details

Mass: inspired by fractals and the visual correspondence of music objects (the
uneven, irregular surface of music notes vs smooth, regular surface). A
measurement for each level in the hierachy. Equals to
∑duration2+pitchInterval2−−−−−−−−−−−−−−−−−−−−−−√. Intuitively, it's the sum of
the lengths of the holding notes and the lengths of the lines connecting the two
notes when there is a pitch change.

Fractal/self-similarity dimension: taking ratios between different levels of
mass, summarising the hierachical structure


\paragraph{Musical features}
Musical features refers to summarising music events numerically. There are
available toolbox to calculate features, such as jMIR \cite{}, MIRtoolbox \cite{}, the FANTASTIC
toolbox \cite{}. There are features such as.  One can either
take the whole piece or take a series of sliding windows and obtain a time
series of features. 

\paragraph{Pattern discovery and classification based on features}
Musical pattern discovery is an active area of research. It faces many
challenges \cite{}. By giving a known hierachy in the piece, we can calculate the 

\paragraph{Our Methods}
We use fractal geometry 

\paragraph{Contributions}
- Based on fractal geometry and the hierachical structures in music, we propose a
new feature that measures the complexity of melodic contours and polyphony
shapes in symbolic music.

- Using the proposed feature, we present a toolset for music analysis and
pattern discovery.

- We showcase the effectiveness of our system on various
corpora and comparing the proposed feature with other existing features of music. 

\section{The similarity dimension}
"Fractal Dimension": $$-D=\frac{logM}{logs}$$

 The Definition of "Mass": $$M\propto{s}^{-D}$$

 \paragraph{Compute the features}
 1. split the music entry into m parts, n bars per part
 
 2. perform the following actions for each bar
 
 1) Create hierachy:
 
      1. take the notes in the most important positions in the bar (for example,
      in a 4/4 bar, we have a importance grid of [5,2,3,2,4,2,3,2] in the
      resolution of a quiver; so only the notes on position of the first quiver
      will be taken)
      
      2. take the notes in the most and the second most important positions in
      the bar (we have the positions of the first and the fifth quiver in this
      case)
      
      3. repeat till we consider all the importance levels
      
 2) Compute measurement (mass) on the hierachy
      
      1. Calculate the mass within one note: = duration in quarter length
 
      2. Calculate the mass between two notes =  $\sum \sqrt{\Delta duration^2 +
        \Delta pitch^2}$ (eqv to the hypotenuse of the time and frequency
      difference)
      
      3. Sum up the mass (intuitively as the length of the line tracing through
      the notes in considerations)
      
   3) Take ratios and the log of the mass between the selected two hierachies: $dim = log_2(mass_{I1}/mass_{I2}) $

\paragraph{Interpretting the feature}
The feature consists of information from two dimensions, time and pitch. We use
a few prototipical example note combinations to illustrate how the fractal
dimensions could reflect the changes in music.

\paragraph{The similarity dimension on one piece}

\section{The Fragem package}
The implementation of the tool is in a functional programming language, Haskell.

\paragraph{From modelling music using data types}
Model of music: [Time Signiture, [Voice]]

Model of mass: [Note] -> Maybe Double
The types give much freedom to how we could calculate the "mass". We choose the length for now for the corresponding visual contours in music. For polyphony, we can extend this to the area enclosed by two voices, and it can capture the amount of contrary motions in the piece, which is crucial for counterpoint.

Model of metrical weights: TimeSig -> [Int]
For each time signiture, we assign a list of integers of importance values to the positions of notes. Now we have a quiver as the resolution of the grid of the positions. New time signitures can be added and the resolution can be changed.

Model of computation: midi -> parameters -> [[Double], [midi]]
The input of frahem is midi files. From the parameters we introduced above, we can specify on which time scale and how many levels of hierachies we would like to analyse. The output is a time series of the fractal/self-similarity dimension. Based on the dimensions, we can also generate the patterns in the midi format with the same dimensions or up to a threshold.

Types of patterns: threshold

\paragraph{parameters}
Zoom level: 1 -> consider all notes, 2 -> consider notes with weights >= 2, …
Window size: how many bars are included in the analysis to produce one number
Sliding or hopping windows
Threshold: what is the maximum gaps between the two groups for them to be considered as belong to the same kind of pattern

@Victor?

\section{Experiment setting}
\subsection{Data}
\paragraph{Synthesised Data}

\paragraph{Hanon}

\paragraph{Bach}

\subsection{Correlation with known features}
Using PCA, we can decompose 

\subsection{Classification}
We use the synthesised data with two levels of randomness, the Hanon exercises
and Bach's fugues for the classification experiment.

Examining the heatmap, we see higher fractal dimension in the most complicated
piece, Bach's WTC. 

\subsection{Pattern discovery}
Using the MIREX dataaset, we found more patterns than the annotations. But
becuase t=if the

\section{Results}

\subsection{Classification}
\subsection{Pattern discovery}


\section{Discussion}

Summary. 

Limitations: 

Future work: Polyphonic


%% Acknowledgments
\begin{acks}                            %% acks environment is optional
                                        %% contents suppressed with 'anonymous'
  %% Commands \grantsponsor{<sponsorID>}{<name>}{<url>} and
  %% \grantnum[<url>]{<sponsorID>}{<number>} should be used to
  %% acknowledge financial support and will be used by metadata
  %% extraction tools.
  This material is based upon work supported by the
  \grantsponsor{GS100000001}{National Science
    Foundation}{http://dx.doi.org/10.13039/100000001} under Grant
  No.~\grantnum{GS100000001}{nnnnnnn} and Grant
  No.~\grantnum{GS100000001}{mmmmmmm}.  Any opinions, findings, and
  conclusions or recommendations expressed in this material are those
  of the author and do not necessarily reflect the views of the
  National Science Foundation.
\end{acks}


%% Bibliography
%\bibliography{bibfile}


%% Appendix
\appendix
\section{Appendix}

Text of appendix \ldots

\end{document}
